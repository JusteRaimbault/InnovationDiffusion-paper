%%%%%%%%%%%%%%%%%%%%%%%%%%%%%
% Standard header for working papers
%
% WPHeader.tex
%
%%%%%%%%%%%%%%%%%%%%%%%%%%%%%

\documentclass[11pt]{article}



%%%%%%%%%%%%%%%%%%%%%%%%%%
%% TEMPLATES
%%%%%%%%%%%%%%%%%%%%%%%%%%


% Simple Tabular

%\begin{tabular}{ |c|c|c| } 
% \hline
% cell1 & cell2 & cell3 \\ 
% cell4 & cell5 & cell6 \\ 
% cell7 & cell8 & cell9 \\ 
% \hline
%\end{tabular}





%%%%%%%%%%%%%%%%%%%%%%%%%%
%% Packages
%%%%%%%%%%%%%%%%%%%%%%%%%%



% encoding 
\usepackage[utf8]{inputenc}
\usepackage[T1]{fontenc}


% general packages without options
\usepackage{amsmath,amssymb,amsthm,bbm}

% graphics
\usepackage{graphicx,transparent,eso-pic}

% text formatting
\usepackage[document]{ragged2e}
\usepackage{pagecolor,color}
%\usepackage{ulem}
\usepackage{soul}


% conditions
\usepackage{ifthen}





%%%%%%%%%%%%%%%%%%%%%%%%%%
%% Maths environment
%%%%%%%%%%%%%%%%%%%%%%%%%%

%\newtheorem{theorem}{Theorem}[section]
%\newtheorem{lemma}[theorem]{Lemma}
%\newtheorem{proposition}[theorem]{Proposition}
%\newtheorem{corollary}[theorem]{Corollary}

%\newenvironment{proof}[1][Proof]{\begin{trivlist}
%\item[\hskip \labelsep {\bfseries #1}]}{\end{trivlist}}
%\newenvironment{definition}[1][Definition]{\begin{trivlist}
%\item[\hskip \labelsep {\bfseries #1}]}{\end{trivlist}}
%\newenvironment{example}[1][Example]{\begin{trivlist}
%\item[\hskip \labelsep {\bfseries #1}]}{\end{trivlist}}
%\newenvironment{remark}[1][Remark]{\begin{trivlist}
%\item[\hskip \labelsep {\bfseries #1}]}{\end{trivlist}}

%\newcommand{\qed}{\nobreak \ifvmode \relax \else
%      \ifdim\lastskip<1.5em \hskip-\lastskip
%      \hskip1.5em plus0em minus0.5em \fi \nobreak
%      \vrule height0.75em width0.5em depth0.25em\fi}



%% Commands

\newcommand{\noun}[1]{\textsc{#1}}


%% Math

% Operators
\DeclareMathOperator{\Cov}{Cov}
\DeclareMathOperator{\Var}{Var}
\DeclareMathOperator{\E}{\mathbb{E}}
\DeclareMathOperator{\Proba}{\mathbb{P}}

\newcommand{\Covb}[2]{\ensuremath{\Cov\!\left[#1,#2\right]}}
\newcommand{\Eb}[1]{\ensuremath{\E\!\left[#1\right]}}
\newcommand{\Pb}[1]{\ensuremath{\Proba\!\left[#1\right]}}
\newcommand{\Varb}[1]{\ensuremath{\Var\!\left[#1\right]}}

% norm
\newcommand{\norm}[1]{\left\lVert #1 \right\rVert}



% argmin
\DeclareMathOperator*{\argmin}{\arg\!\min}


% amsthm environments
\newtheorem{definition}{Definition}
\newtheorem{proposition}{Proposition}
\newtheorem{assumption}{Assumption}

%% graphics

% renew graphics command for relative path providment only ?
%\renewcommand{\includegraphics[]{}}








% geometry
\usepackage[margin=2cm]{geometry}

% layout : use fancyhdr package
\usepackage{fancyhdr}
\pagestyle{fancy}

\makeatletter

\renewcommand{\headrulewidth}{0.4pt}
\renewcommand{\footrulewidth}{0.4pt}
\fancyhead[RO,RE]{\textit{Working Paper}}
\fancyhead[LO,LE]{ISC-PIF}
\fancyfoot[RO,RE] {\thepage}
\fancyfoot[LO,LE] {\noun{J. Raimbault}}
\fancyfoot[CO,CE] {}
\usepackage{natbib}
\usepackage[colorlinks=true, linkcolor=red, citecolor = blue, urlcolor = magenta]{hyperref}
\makeatother


%%%%%%%%%%%%%%%%%%%%%
%% Begin doc
%%%%%%%%%%%%%%%%%%%%%







\title{An empirical study of the diffusion of innovation within systems of cities through the lens of patent data\\
\textit{Project}
}
\author{}
\date{Date}


\maketitle

\justify


\begin{abstract}

\end{abstract}




%%%%%%%%%
\section{Introduction}

Following \cite{hagerstrand1968innovation}, the evolutive urban theory considers innovation waves between cities as a crucial components of their interactions. \cite{favaro2011gibrat} integrate this view into a model of growth for a system of cities.

Empirical quantifications at a large scale and on long time scales are however rare in the literature. \cite{grubler1996time} shows on examples a typical logistic curve of adoption rate in time. \cite{albuquerque2007spatiotemporal} investigate the diffusion of two norms across countries.

\cite{acs2002patents} uses patents to study regional productivity.

\cite{wejnert2002integrating} framework for integrated models

\cite{sonis1983spatio} Lotka-Volterra model for competing innovations

We propose to study the diffusion of innovation in urban systems through the lens of exhaustive patent data, combining different layers of complete patents networks.

This contribution is new on the following points: (i) we are to the best of our knowledge the first to use exhaustive patents data sets; (ii) we take an urban system perspective, more precisely in the context of the evolutive urban theory; (iii)


%%%%%%%%%
\section{Project}

\subsection{Objective}

\begin{itemize}
	\item Existence of spatio-temporal patterns in the different layers of patents network ?
	\item Patterns depending of types of techno ?
	\item In this case, relation of these patterns to the dynamics of the urban system ?
\end{itemize}



\subsection{Protocol}


\begin{enumerate}
	\item Formalize the theoretical framework and assumptions to be tested
	\item Consolidate databases
	\item Exploratory analysis
	\item Targeted analysis (specific questions): modeling ?
\end{enumerate}



%%%%%%%%%
\section{Details}


\subsection{Datasets}

\begin{enumerate}
	\item USPTO : citation and semantic classification 1976-2012 \cite{bergeaud2017classifying} ; localisation of inventors / or establishments (more accurate ?) \cite{li2014disambiguation}
	\item European patent data : clean database ? (\textit{semantic processing to be done}) ; which localization ?
	\item Cities database (spatial granularity: long time ontology for cities - evolving boundaries \cite{bretagnolle2009villes}) : ERC Geodivercity
	\item \textit{Chinese and japanese patents ? future perspective}
\end{enumerate}


\subsection{Methods}

\subsubsection{Quantifying diffusion processes}

\begin{itemize}
	\item Spatio-temporal econometrics
	\item Diffusion models (networks, ecology, reaction-diffusion)
\end{itemize}








%%%%%%%%%%%%%%%%%%%%
%% Biblio
%%%%%%%%%%%%%%%%%%%%

\bibliographystyle{apalike}
\bibliography{../../Biblio/innovationDiffusion}


\end{document}
